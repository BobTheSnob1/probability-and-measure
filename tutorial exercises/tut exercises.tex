\documentclass{article}
\usepackage[utf8]{inputenc}
\usepackage{amsmath, amssymb, amsthm}

\title{Tutorial Exercises: Probability and Measure}
\author{Richard Harnisch (S5238366)}
\date{\today}

\begin{document}

\maketitle
\sloppy

\section*{Tutorial 1}
\subsection*{Exercise 1}

To show that $\mathcal{A}$ is a $\sigma$-algebra, we need to verify three properties:
\begin{itemize}
    \item Non-empty: $\Omega \in \mathcal{A}$ \\
    $\Omega$ is a subset of itself. Additionally, $\Omega^c = \emptyset$, which is countable. Thus $\Omega \in \mathcal{A}$.

    \item Closed under complement: $A \in \mathcal{A} \implies A^c \in \mathcal{A}$ \\
    First note that ${A^c}^c = A$. If $A \in \mathcal{A}$, then either $A$ or $A^c$ is countable. Now consider $A^c$: It is in $\mathcal{A}$ if $A^c$ or ${A^c}^c$, which is $A$, is countable. We know this must be true. Therefore, $A \in \mathcal{A} \implies A^c \in \mathcal{A}$. 

    \item Closed under countable unions: $A_1, A_2, \dots, A_n \in \mathcal{A} \implies \bigcup_{i=1}^{n} A_i \in \mathcal{A}$ \\
    For the purpose of notation, $\bigcup_{i=1}^{n} A_i := U$. We will prove this by examining two exhaustive cases. \\
    \textbf{Case 1:} There exists an $A^c_k$ that is countable. \\
    By DeMorgan's law, $U^c = \bigcap_{i=1}^{n} A^c_i$. Obviously, $\bigcap_{i=1}^{n} A^c_i \subset A^c_k$. Since $A^c_k$ is countable, $U^c$ is too and is in $\mathcal{A}$. \\
    \textbf{Case 2:} There is no $A^c_k$ that is countable. \\
    Then, every $A^c$ must be countable to be in $\mathcal{A}$. Thus $U$ is the union of a countable collection of countable sets and is therefore itself countable. Therefore $U \in \mathcal{A}$.
\end{itemize}
\pagebreak
\subsection*{Exercise 2}
\subsubsection*{(a)}
This set is the intersection of all $A_n$. 
Consider the set of all $A_n$. Each $A^c_n$ is in $\mathcal{A}$ by property 2 of $\sigma$-algebras, and so is their union by property three. The complement of this union must thus also be by property two, which is equal to the intersection of all $A_n$ by applying DeMorgan's law. \qed

\subsubsection*{(b)}
This set is the intersection of all $A^c_n$. 
Consider the set of all $A^c_n$. Each ${A^c_n}^c = A$ is in $\mathcal{A}$, and so is their union by property three of $\sigma$-algebras. The complement of this union must thus also be by property two, which is equal to the intersection of all $A^c_n$. \qed

\subsubsection*{(c)}
This is the intersection of an infinite set of $A_n$. The same reasoning as for (a) holds. 

\subsubsection*{(d)}
This is the intersection of an infinite set of $A_n$. As in (c), the same reasoning as for (a) holds. 

\subsection*{Exercise 3}
\begin{align*}
    \Omega &= \mathbb{N} \\
    A &= \{\emptyset, \{1\}, \{1\}^c, \mathbb{N}\} \\
    B &= \{\emptyset, \{2\}, \{2\}^c, \mathbb{N}\} \\
    \{1\} \cup \{2\} &= \{1, 2\} \notin A \cup B
\end{align*}

$A$ and $B$ are $\sigma$-algebras, but $A \cup B$ is not by violating clusure under countable unions.

\subsection*{Exercise 4}

To show $\mu$ is a measure on $\Omega$, we need to verify three properties:
\begin{itemize}
    \item \textbf{Nonnegativity:} Any $p_n$ is a nonnegative number. Thus the sum of any collection of such $p_n$ will also be nonnegative.
    \item \textbf{Null empty set:} $\mu(\emptyset) = 0$ because it is a sum over no summands.
    \item \textbf{Countable additivity:} $A_1, A_2, \dots \subset \mathbb{N}$ are pairwise disjoint. Since $\mu(A_i)$ is the sum of all $p_n, n \in A_i$, and all $A_n$ are disjoint, if we take the $\mu$ of a union of $A$s, then we are simply concatenating sequences of nonnegative numbers. Taking the sum of the whole sequence or taking it piecewise and then the sum of the pieces does not change the result. Therefore $\mu$ satisfies countable additivity.
\end{itemize}

\subsection*{Exercise 5}
\begin{itemize}
    \item $\Omega \in \mathcal{A}$
    \item $A \in \mathcal{A} \implies A^c \in \mathcal{A}$
    \item $A, B \in \mathcal{A}$ disjoint $\implies A \cup B \in \mathcal{A}$
\end{itemize}
Show:
\[
    A, B \in \mathcal{A}, A \subset B \implies B \backslash A \in A
\]

\begin{align*}
    B^c &\in \mathcal{A} \\
    B^c \cap A &= \emptyset \\
    B^c \cup A &\in \mathcal{A} \\
    (B^c \cup A)^c &= B \backslash A \\
    B \backslash A &\in \mathcal{A} \\
    \qed
\end{align*}

\section*{Tutorial 2}
\subsection*{Exercise 1}
\[
    \mathbb{P}_B = \mathbb{P}(A|B) = \frac{\mathbb{P}(A \cap B)}{\mathbb{P}(B)}
\]

\subsubsection*{(a)}

To show $\mathbb{P}_B$ is a probability measure, we need to show non-negativity, countable additivity and that the total mass $\mathbb{P}(\Omega) = 1$.

\begin{enumerate}
    \item \textbf{Non-negativity:} Take any $A \in \mathcal{A}$. Consider $\mathbb{P}_B (A) = \frac{\mathbb{P}(A \cap B)}{\mathbb{P}(B)}$. Since $\sigma$-algebras are closed under intersections, $A \cap B \in \mathcal{A}$, so $\mathbb{P}(A\cap B) \geq 0$ since $\mathbb{P}$ is a probability measure. Since $\mathbb{P}(B) \geq 0$, both the numerator and the denominator of $\frac{\mathbb{P}(A \cap B)}{\mathbb{P}(B)}$ are $\geq 0$ and so the whole fraction is too. Therefore $\forall A \in \mathcal{A}: \mathbb{P}_B (A) \geq 0$.

    \item \textbf{Countable additivity:} Let $(A_i)_{i \geq 1} \subset \mathcal{A}$ be pairwise disjoint, e.g. they share no elements. We need to show:
    \[
        \mathbb{P}_B (\bigcup_{i \geq 1} A_i) = \sum_{i \geq 1} \mathbb{P}_B (A_i)
    \]
    We can consider the left-hand side of this equation:
    \[
        \mathbb{P}_B (\bigcup_{i \geq 1} A_i) = \frac{\mathbb{P}((\bigcup_i A_i) \cap B)}{\mathbb{P}(B)} = \frac{\mathbb{P}(\bigcup_i (A_i \cap B))}{\mathbb{P}(B)}
    \]
    Recall that $\mathbb{P}$ is countably additive. Therefore:
    \[
        \frac{\mathbb{P}(\bigcup_i (A_i \cap B))}{\mathbb{P}(B)} = \frac{\sum_{i \geq 1} \mathbb{P}(A_i \cap B)}{\mathbb{P}(B)} = \sum_{i \geq 1} \frac{\mathbb{P}(A_i \cap B)}{\mathbb{P}(B)} = \sum_{i \geq 1} \mathbb{P}_B (A_i)
    \]
    This is exactly the right-hand side. Thus $\mathbb{P}_B$ is countably additive.

    \item \textbf{Normalization ($\mathbb{P}(\Omega) = 1$)}:
    \[
        \mathbb{P}(\Omega) = \frac{\mathbb{P}(\Omega \cap B)}{\mathbb{P}(B)} = \frac{\mathbb{P}(B)}{\mathbb{P}(B)} = 1
    \]

\end{enumerate}

\subsubsection*{(b)}

\[
    ((\mathbb{P}_{B_1})_{B_2})(A) = \mathbb{P}_{B_1} (A|B_2) = \frac{\mathbb{P}_{B_1}(A\cap B_2)}{\mathbb{P}_{B_1}(B_2)} = \frac{\mathbb{P}((A\cap B_2) | B_1)}{\mathbb{P}(B_2 | B_1)}
\]
\[
    = \frac{
        \frac{\mathbb{P}((A\cap B_2) \cap B_1)}{\mathbb{P}(B_1)}
    }{
        \frac{\mathbb{P}(B_2 \cap B_1)}{\mathbb{P}(B_1)}
    }
\]
\[
    = \frac{\mathbb{P}((A\cap B_2) \cap B_1)}{\mathbb{P}(B_2 \cap B_1)} = \frac{\mathbb{P}(A \cap (B_2 \cap B_1))}{\mathbb{P}(B_2 \cap B_1)} = \mathbb{P}_{B_2 \cap B_1}(A) = \mathbb{P}_{B_1 \cap B_2}(A)
\]

\subsection*{Exercise 2}

Two events $A, B$ are called \textbf{independent} if
\[
    \mathbb{P}(A \cap B) = \mathbb{P}(A) \cdot \mathbb{P}(B)
\]

\subsubsection*{(a)}
$A, B$ are independent.
    \begin{itemize}
        \item $A, B^c$: Use the fact that $A$ is the disjoint union $A = (A \cap B^c) \cup (A \cap B)$. Therefore:
        \begin{align*}
            \mathbb{P}(A) &= \mathbb{P}(A \cap B^c) + \mathbb{P}(A \cap B) \\
            \mathbb{P}(A) &= \mathbb{P}(A \cap B^c) + \mathbb{P}(A)\cdot\mathbb{P}(B) \\
            \mathbb{P}(A) - \mathbb{P}(A)\cdot\mathbb{P}(B) &= \mathbb{P}(A \cap B^c) \\
            \mathbb{P}(A)(1 - \mathbb{P}(B)) &= \mathbb{P}(A \cap B^c) \\
            \mathbb{P}(A)\cdot\mathbb{P}(B^c) &= \mathbb{P}(A \cap B^c) \\ 
            \qed
        \end{align*}
        \item $A^c, B$: Use the exact same reasoning as above:
        \begin{align*}
            B &= (B \cap A^c) \cup (B \cap A) \\
            \mathbb{P}(B) &= \mathbb{P}(B \cap A^c) + \mathbb{P}(B \cap A) \\
            \mathbb{P}(B) &= \mathbb{P}(B \cap A^c) + \mathbb{P}(B)\cdot\mathbb{P}(A) \\
            \mathbb{P}(B) - \mathbb{P}(B)\cdot\mathbb{P}(A) &= \mathbb{P}(B \cap A^c) \\
            \mathbb{P}(B)(1 - \mathbb{P}(A)) &= \mathbb{P}(B \cap A^c) \\
            \mathbb{P}(B)\cdot\mathbb{P}(A^c) &= \mathbb{P}(B \cap A^c) \\ 
            \qed
        \end{align*}
        \item $A^c, B^c$: We need to show:
        \begin{equation}
            \mathbb{P}(A^c \cap B^c) = \mathbb{P}(A^c)\cdot\mathbb{P}(B^c)
        \end{equation}
        Use the fact that $A^c \cap B^c = (A \cup B)^c$:
        \begin{align*}
            \mathbb{P}(A^c \cap B^c) &= \mathbb{P}((A \cup B)^c) \\
            \mathbb{P}(A^c \cap B^c) &= 1 - \mathbb{P}(A \cup B)
        \end{align*}
        Recall that $\mathbb{P}(A \cup B) = \mathbb{P}(A) + \mathbb{P}(B) - \mathbb{P}(A \cap B)$:
        \begin{align*}
            \mathbb{P}(A^c \cap B^c) &= 1 - (\mathbb{P}(A) + \mathbb{P}(B) - \mathbb{P}(A \cap B)) \\
            \mathbb{P}(A^c \cap B^c) &= 1 - \mathbb{P}(A) - \mathbb{P}(B) + \mathbb{P}(A)\cdot\mathbb{P}(B) \\
            \mathbb{P}(A^c \cap B^c) &= (1 - \mathbb{P}(A))(1 - \mathbb{P}(B)) \\
            \mathbb{P}(A^c \cap B^c) &= \mathbb{P}(A^c)\cdot\mathbb{P}(B^c) \\
            \qed
        \end{align*}
    \end{itemize}

\section*{Tutorial 3}
\subsection*{Exercise 1}
The $\sigma$-algebra generated by the $\{\omega\}$ sets is the smallest possible $\sigma$-algebra containing all of the $\{\omega\}$ sets. By the properties of $\sigma$-algebras, this means it contains all countable subsets of $\Omega$ and their complements. It includes all countable subsets of $\Omega$ because $\{\omega\}$ are the singleton sets of all elements in $\Omega$. Using property 3 of $\sigma$-algebras, we can construct arbitrarily large countable sets containing any number of elements from $\Omega$, which must be in the algebra. Using property 2, their complements must also be in it. This is precisely the definition of the collection, so the algebra is equal to the collection.

\section*{Tutorial 4}
\subsection*{Exercise 1}
To show that $\mu^*$ is an outer measure, we need to verify three properties:
\begin{itemize}
    \item \textbf{Null empty set:} Given by definition
    \item \textbf{Monotonicity:} We need to show that $A \subset B \implies \mu^*(A) \leq \mu^*(B)$. By looking at the definition of $\mu^*$, we can see that as the size of A is increasing, the outer measure is monotonically increasing. This necesarily gives monotonicity, since any subset of A will have lower or equal size and thus lower or equal outer measure.
    \item \textbf{Sub-additivity:} Consider three cases:
        \begin{itemize}
            \item The union is the empty set: In this case each element is also the empty set and the outer measure of the union is zero and the outer measuer of each of its elements (empty sets) is also zero, so their sum is also zero. $0 \leq 0$.
            \item The union has one or two elements: This means that the union has outer measure one and at least one of the individual sets has one or two members, which means that its  outer measure is also one. Therefore the sum of the outer measures must be at least one, which is equal or more than the outer measure of the union
            \item The union has all three elements, so the outer measure is at least 2. This means that either $\Omega$ is in the set of sets and therefore the the sum of the outer measures is also at least two, or it means that there are at least two nonempty sets whose union contains all three elements, each of which has outer measure of one. Their sum is then at least two. This means the sum of the outer measures of the individual sets is at least two. This means the sum is at least the outer measure of the union.
        \end{itemize}
\end{itemize}

\subsection*{Exercise 2}
By monotonicity of $\mu^*$:
\[
    B \subset B \cup A \implies \mu^*(B) \leq \mu^*(B \cup A)
\]

By sub-additivity of $\mu^*$:
\[
    \mu^*(A \cup B) \leq \mu^*(A) + \mu^*(B) = 0 + \mu^*(B) = \mu^*(B)
\]

Thus:
\begin{align*}
    \mu^*(B) \leq \mu^*(B \cup A) \leq \mu^*(B) \\
    \therefore \mu^*(B) = \mu^*(B \cup A) \qed
\end{align*}

\subsection*{Exercise 3}
?????????? \\
\textit{B is a superset of A, and A is measurable. Additionally, the outer measures of B and A are the same and finite. A is measurable if there exists a finite mu(A). Since B has a finite outer measure and the measure must always be smaller or equal to the outer measure, B must also be measurable}

\end{document}